\documentclass[uplatex,dvipdfmx,ja=standard]{bxjsarticle}
\begin{document}

\newcommand{\typing}[3]{#1 $\vdash$ #2{\tt:}#3}

\newcommand{\varx}{{\tt x}}
\newcommand{\typet}{{\tt T}}
\newcommand{\typeu}{{\tt U}}
\newcommand{\tybool}{{\tt U}}
\newcommand{\tynat}{{\tt Nat}}
\newcommand{\ctxtg}{{$\Gamma$}}
\newcommand{\funarrow}{{$\rightarrow$}}

存在しない。

型の深さを次で定義する。
\begin{itemize}
\item \tybool の深さは0である。
\item \tynat の深さは0である。
\item \typet \funarrow \typeu の深さは
		\typet の深さと \typeu の深さの大きい方(小さくない方)に1を加えた値である。
\end{itemize}

定義より型の深さは一意である。

いま、ある文脈\ctxtg 、 変数\varx 、 型\typet について \typing{\ctxtg}{\varx \varx}{\typet} であるとする。
補題9.3.1よりある型\typeu が存在して
\typing{\ctxtg}{\varx}{\typeu \funarrow \typet}および
\typing{\ctxtg}{\varx}{\typeu}
が成り立つ。
\varx は変数であるので、
\ctxtg(\varx) = \typeu = \typeu \funarrow \typet
であるが、
\typeu \funarrow \typet の深さは \typeu の深さより常に大きい。これは型の深さが一意であることに反する。

\end{document}
